% Options for packages loaded elsewhere
% Options for packages loaded elsewhere
\PassOptionsToPackage{unicode}{hyperref}
\PassOptionsToPackage{hyphens}{url}
%
\documentclass[
  ignorenonframetext,
  aspectratio=169,
]{beamer}
\newif\ifbibliography
\usepackage{pgfpages}
\setbeamertemplate{caption}[numbered]
\setbeamertemplate{caption label separator}{: }
\setbeamercolor{caption name}{fg=normal text.fg}
\beamertemplatenavigationsymbolsempty
% remove section numbering
\setbeamertemplate{part page}{
  \centering
  \begin{beamercolorbox}[sep=16pt,center]{part title}
    \usebeamerfont{part title}\insertpart\par
  \end{beamercolorbox}
}
\setbeamertemplate{section page}{
  \centering
  \begin{beamercolorbox}[sep=12pt,center]{section title}
    \usebeamerfont{section title}\insertsection\par
  \end{beamercolorbox}
}
\setbeamertemplate{subsection page}{
  \centering
  \begin{beamercolorbox}[sep=8pt,center]{subsection title}
    \usebeamerfont{subsection title}\insertsubsection\par
  \end{beamercolorbox}
}
% Prevent slide breaks in the middle of a paragraph
\widowpenalties 1 10000
\raggedbottom
\AtBeginPart{
  \frame{\partpage}
}
\AtBeginSection{
  \ifbibliography
  \else
    \frame{\sectionpage}
  \fi
}
\AtBeginSubsection{
  \frame{\subsectionpage}
}
\usepackage{iftex}
\ifPDFTeX
  \usepackage[T1]{fontenc}
  \usepackage[utf8]{inputenc}
  \usepackage{textcomp} % provide euro and other symbols
\else % if luatex or xetex
  \usepackage{unicode-math} % this also loads fontspec
  \defaultfontfeatures{Scale=MatchLowercase}
  \defaultfontfeatures[\rmfamily]{Ligatures=TeX,Scale=1}
\fi
\usepackage{lmodern}

\usetheme[]{default}
\ifPDFTeX\else
  % xetex/luatex font selection
\fi
% Use upquote if available, for straight quotes in verbatim environments
\IfFileExists{upquote.sty}{\usepackage{upquote}}{}
\IfFileExists{microtype.sty}{% use microtype if available
  \usepackage[]{microtype}
  \UseMicrotypeSet[protrusion]{basicmath} % disable protrusion for tt fonts
}{}
\makeatletter
\@ifundefined{KOMAClassName}{% if non-KOMA class
  \IfFileExists{parskip.sty}{%
    \usepackage{parskip}
  }{% else
    \setlength{\parindent}{0pt}
    \setlength{\parskip}{6pt plus 2pt minus 1pt}}
}{% if KOMA class
  \KOMAoptions{parskip=half}}
\makeatother


\usepackage{longtable,booktabs,array}
\usepackage{calc} % for calculating minipage widths
\usepackage{caption}
% Make caption package work with longtable
\makeatletter
\def\fnum@table{\tablename~\thetable}
\makeatother
\usepackage{graphicx}
\makeatletter
\newsavebox\pandoc@box
\newcommand*\pandocbounded[1]{% scales image to fit in text height/width
  \sbox\pandoc@box{#1}%
  \Gscale@div\@tempa{\textheight}{\dimexpr\ht\pandoc@box+\dp\pandoc@box\relax}%
  \Gscale@div\@tempb{\linewidth}{\wd\pandoc@box}%
  \ifdim\@tempb\p@<\@tempa\p@\let\@tempa\@tempb\fi% select the smaller of both
  \ifdim\@tempa\p@<\p@\scalebox{\@tempa}{\usebox\pandoc@box}%
  \else\usebox{\pandoc@box}%
  \fi%
}
% Set default figure placement to htbp
\def\fps@figure{htbp}
\makeatother





\setlength{\emergencystretch}{3em} % prevent overfull lines

\providecommand{\tightlist}{%
  \setlength{\itemsep}{0pt}\setlength{\parskip}{0pt}}



 
\usepackage[style=apa,backend=biber]{biblatex}
\addbibresource{references.bib}


\usepackage{booktabs}
\usepackage{longtable}
\usepackage{array}
\usepackage{multirow}
\usepackage{wrapfig}
\usepackage{float}
\usepackage{colortbl}
\usepackage{pdflscape}
\usepackage{tabu}
\usepackage{threeparttable}
\usepackage{threeparttablex}
\usepackage[normalem]{ulem}
\usepackage{makecell}
\usepackage{xcolor}
\usepackage{tabularray}
\usepackage[normalem]{ulem}
\usepackage{graphicx}
\usepackage{rotating}
\UseTblrLibrary{siunitx}
\NewTableCommand{\tinytableDefineColor}[3]{\definecolor{#1}{#2}{#3}}
\newcommand{\tinytableTabularrayUnderline}[1]{\underline{#1}}
\newcommand{\tinytableTabularrayStrikeout}[1]{\sout{#1}}
\usepackage{tcolorbox}
  \tcbset{
   colback=blue!5,
   colframe=blue!50!black,
   boxrule=0.4pt,
   arc=2pt,
   left=4pt,
   right=4pt,
   top=3pt,
   bottom=3pt
 }
\usepackage{lmodern}
\usefonttheme{serif}
\setbeamercolor{structure}{fg=black}
\setbeamerfont{frametitle}{series=\bfseries}
\setbeamerfont{section title}{series=\bfseries}
\setbeamertemplate{itemize item}{\textbullet}
\setbeamertemplate{itemize subitem}{\textbullet}
\setbeamertemplate{itemize subsubitem}{\textbullet}
\usepackage{multicol}
\tcolorboxenvironment{block}{}
\tcolorboxenvironment{alertblock}{}
\tcolorboxenvironment{exampleblock}{}
\usepackage{amssymb}
\makeatletter
\@ifpackageloaded{caption}{}{\usepackage{caption}}
\AtBeginDocument{%
\ifdefined\contentsname
  \renewcommand*\contentsname{Table of contents}
\else
  \newcommand\contentsname{Table of contents}
\fi
\ifdefined\listfigurename
  \renewcommand*\listfigurename{List of Figures}
\else
  \newcommand\listfigurename{List of Figures}
\fi
\ifdefined\listtablename
  \renewcommand*\listtablename{List of Tables}
\else
  \newcommand\listtablename{List of Tables}
\fi
\ifdefined\figurename
  \renewcommand*\figurename{Figure}
\else
  \newcommand\figurename{Figure}
\fi
\ifdefined\tablename
  \renewcommand*\tablename{Table}
\else
  \newcommand\tablename{Table}
\fi
}
\@ifpackageloaded{float}{}{\usepackage{float}}
\floatstyle{ruled}
\@ifundefined{c@chapter}{\newfloat{codelisting}{h}{lop}}{\newfloat{codelisting}{h}{lop}[chapter]}
\floatname{codelisting}{Listing}
\newcommand*\listoflistings{\listof{codelisting}{List of Listings}}
\makeatother
\makeatletter
\makeatother
\makeatletter
\@ifpackageloaded{caption}{}{\usepackage{caption}}
\@ifpackageloaded{subcaption}{}{\usepackage{subcaption}}
\makeatother

\usepackage{bookmark}
\IfFileExists{xurl.sty}{\usepackage{xurl}}{} % add URL line breaks if available
\urlstyle{same}
\hypersetup{
  pdftitle={Availability of Abitur and Non-Abitur Upper Secondary Schools and Housing Prices in NRW},
  hidelinks,
  pdfcreator={LaTeX via pandoc}}


\title{Availability of Abitur and Non-Abitur Upper Secondary Schools and
Housing Prices in NRW}
\author{Julian Schilling \and Benedikt Teuber \and Elena Preussner}
\date{01.02.2026}

\begin{document}
\frame{\titlepage}

\renewcommand*\contentsname{Table of contents}
\begin{frame}[allowframebreaks]
  \frametitle{Table of contents}
  \setcounter{tocdepth}{3}
  \tableofcontents
\end{frame}
\setcounter{tocdepth}{3}
\tableofcontents
}

\section{Introduction}\label{introduction}

\begin{frame}{Research Question}
\phantomsection\label{research-question}
\begin{block}{Main question}
To what extent does the local availability of upper secondary schools that offer an Abitur pathway, compared to other secondary schools, affect housing prices in North Rhine-Westphalia (NRW)?
\end{block}

\begin{itemize}
\tightlist
\item
  Goal: Estimating heterogeneity in the capitalization of secondary
  schools that offer the possibility to achieve higher educational
  outcomes.
\item
  We assume that parents are willing to pay a premium for their homes,
  if a school lies within 3km at which their children can make an Abitur
  (Gymnasium or Gesamtschule)
\item
  Therefore, we restrict our analysis on different types of secondary
  schools and exclude elementary schools and other specialised or
  private schools
\end{itemize}
\end{frame}

\begin{frame}{Motivation and Institutional Context}
\phantomsection\label{motivation-and-institutional-context}
\begin{itemize}
\tightlist
\item
  School availability and quality in general are important determinants
  of housing decisions and are capitalized into housing prices (..)
\item
  However, existing evidence on capitalization differences across upper
  secondary school tracks is limited, while most hedonic price studies
  face endogeneity concerns.
\item
  The
  \textbf{magnitude of capitalization effects may differ by school track}
\item
  Germany's multi-track secondary school system provides a unique
  setting to study differential capitalization effects.
\item
  In NRW the 2008/09 reform increased freedom in school-choice and
  weakens formal residence--school links as it may
  \textbf{increase behavioral selection} into high-quality school areas.
\end{itemize}

\begin{block}{Key Insight:}
    When school choice is flexible, households with strong school preferences are more likely to relocate to access better schools — reinforcing the link between school quality and housing prices. \parencite{bayer2007}
    \end{block}
\end{frame}

\section{Theoretical Framework}\label{theoretical-framework}

\begin{frame}{Tiebout sorting}
\phantomsection\label{tiebout-sorting}
\begin{itemize}
\tightlist
\item
  The decision-making process of residents include the availability and
  quality of provided public goods and services within a municipality
\item
  Under the assumption of perfect mobility, residents pick that
  community that exactly satisfies their preferences
  \parencite{tiebout1956}
\item
  If such a community or municipality is not feasible, a perfect
  substitute (if existent) is to be chosen
\end{itemize}
\end{frame}

\begin{frame}{What do parents value?}
\phantomsection\label{what-do-parents-value}
\begin{itemize}
\tightlist
\item
  Educational quality is an important part of the set of considered
  public goods, because quality schooling is often decisive in later
  life-outcomes (e.g.~labor market opportunities, gained income, health
  etc.)
\item
  The way in which parents sort into the housing market directly
  influencing the level of residentual segregation \parencite{bayer2007}
\item
  Ongoing debate, which dimension of education is valued by parents
  (outputs or learning environments containing sociodemographic
  composition) \parencite{machin2011}
\item
  We want to test whether the availability of upper secondary schools is
  such a dimension
\end{itemize}
\end{frame}

\begin{frame}{Capitalization mechanism}
\phantomsection\label{capitalization-mechanism}
\begin{itemize}
\tightlist
\item
  Parents are willing to pay a premium for housing units nearby top-tier
  school networks \parencite{jayantha2015}
\end{itemize}

This capitalizes into the housing market via two mechanisms
\parencite{la2015}:

\begin{itemize}
\tightlist
\item
  Wealthier households that care about school quality bid up prices
  within the walking zone of a school
\item
  generation of spillovers through changes in neighborhood composition
\end{itemize}
\end{frame}

\begin{frame}{Hypothesis}
\phantomsection\label{hypothesis}
Based on the theoretical background, we expect the following results:

\begin{itemize}
\tightlist
\item
  Education is one of the most important public services
  \parencite{zhang2020} and it is therefore reasonable to test the
  channel
\item
  Educational opportunities play a role in parents' housing decisions
\item
  Parents value the opportunity to achieve higher educational outcomes
  for their children \parencite{hornig2025} because of their
  decisiveness for later life outcomes
\item
  Parents are therefore willing to pay a premium for houses near a
  secondary school offering the opportunity to obtain an \emph{Abitur}
\end{itemize}

\begin{block}{Formal Hypothesis}
\[
\begin{aligned}
\tau_{\text{school}} & \text{(Estimated treatment effect):} & H_1 &: \beta_1 (D_i) > 0 \\
\tau_{\text{abitur}} & \text{(Estimated treatment heterogeneity):} & H_2 &: \beta_2 (D_i \times A_i)  > 0
\end{aligned}
\]
\end{block}
\end{frame}

\section{Insights from relevant empirical
literature}\label{insights-from-relevant-empirical-literature}

\begin{frame}{Literature Insights (excerpt)}
\phantomsection\label{literature-insights-excerpt}
\begin{itemize}
\tightlist
\item
  The empirical literature mainly focused on the capitalization effects
  of school quality
\item
  UK: Strong capitalization of \textbf{primary school} performance into
  housing prices. \parencite{gibbons2003}
\item
  US: Stronger price effects from \textbf{middle and high school}
  quality than from elementary schools.\parencite{sedgley2008}
\item
  France: Secondary school quality capitalized more strongly in areas
  without \textbf{private school alternatives}. \parencite{fack2010}
\end{itemize}
\end{frame}

\section{Empirical Design}\label{empirical-design}

\begin{frame}{Basic Identification Assumptions}
\phantomsection\label{basic-identification-assumptions}
\textbf{Theoretical Assumptions:}

\begin{itemize}
\tightlist
\item
  Parents (Households) derive utility from the perceived quality of
  schooling \parencite{brunner2012} available to their child
\item
  Parents consider the availability of Abitur and non-Abitur upper
  secondary schools in their housing decisions.
\end{itemize}

\textbf{Empirical Assumptions:}

\begin{itemize}
\tightlist
\item
  Parents' preferences for their children's education are reflected in
  hedonic price regressions.
\item
  Conditional on the controlling for both housing and neighborhood
  characteristics, the treatment assignment can considered to be random.
\item
  Within the treatment area and the control area, the capitalization
  effect is constant.
\end{itemize}
\end{frame}

\begin{frame}{Assumptions (II)}
\phantomsection\label{assumptions-ii}
\begin{itemize}
\tightlist
\item
  Buildings inside and outside of the treatment zones share the same
  average housing and neighborhood characteristics.
\item
  Offer prices are time-independently exceeding sale prices at a
  constant rate.
\item
  Property prices can be interpreted as the willingness to pay for
  amenities because they are determined by relevant characteristics
\end{itemize}
\end{frame}

\begin{frame}{Data and Variables}
\phantomsection\label{data-and-variables}
\textbf{Housing Data:}

\begin{itemize}
\tightlist
\item
  Geo-referenced listings of sales properties (Internet platform
  lmmobilienScout24, 2022)
\item
  Variables: living area, site area, number of rooms, number of
  bathrooms, year of construction, cellar
\end{itemize}

\vspace{0.2cm}

\textbf{School Data}:

\begin{itemize}
\tightlist
\item
  Locations and types of schools (Grammar School, Comprehensive School,
  etc. based on grid cells)
\end{itemize}

\vspace{0.2cm}

\textbf{Regional Data:}

\begin{itemize}
\tightlist
\item
  Includes information on neighborhood characteristics (e.g.~income
  levels, migration rates, average age and the availability of
  infrastructure such as doctors or a supermarket)
\end{itemize}
\end{frame}

\begin{frame}{Quasi-experimental approach}
\phantomsection\label{quasi-experimental-approach}
\begin{columns}[T] % top aligned

  % -------- LEFT COLUMN --------
  \begin{column}{0.55\textwidth}
    We define spatial treatment zones to estimate the causal effect of upper secondary schools on housing prices using grid-cells:

    \vspace{0.6em}

    \begin{itemize}
      \item \textbf{Treatment-Zone:} Grid-cells \textbf{within 2 km} radius
      \item \textbf{Untreated zone:} Grid-cells \textbf{2--4 km} from any school
      \item Houses exposed to multiple secondary schools (double-treated) are excluded
    \end{itemize}
  \end{column}

  % -------- RIGHT COLUMN --------
  \begin{column}{0.45\textwidth}
    \centering
    \includegraphics[width=\linewidth]{plot_zones_no_buffer.pdf}
  \end{column}

\end{columns}

\begin{block}{Robustness check}
As a robustness check, we exclude a \textbf{Buffer zone} of 1km around the treatment zone to prevent spillover contamination, therefore reducing the amount of control units
\end{block}
\end{frame}

\begin{frame}{Identification Framework}
\phantomsection\label{identification-framework}
\textbf{Potential Outcome Model (POM):} following \cite{Rubin1974}
\[\ln(P_{ij}) =
\begin{cases}
\ln(P_{1ij}) & \text{if } D_i = 1, \\
\ln(P_{0ij}) & \text{if } D_i = 0.
\end{cases}
\]
\noindent\textbf{where:}
\[
\begin{aligned}
\ln(P_{1ij}) &:\ \text{Price of the house, when it lies in the treatment-zone}\\
              &\quad \text{near to a secondary school}.\\
\ln(P_{0ij}) &:\ \text{Price of the house, when it would not have been 'exposed'}\\
              &\quad \text{to a secondary school (counterfactual)}.
\end{aligned}\]

\begin{block}{Note}
Since the counterfactual is not observed, we use the most similar house
lying in the non-treated zone as a proxy for the counterfactual.
\end{block}
\end{frame}

\begin{frame}{Matching Strategy}
\phantomsection\label{matching-strategy}
Identification Assumption (CIA):
\[ \ln(P_{0ij}), \ln(P_{1ij}) \;\perp\; D \;\big| X_i\]
\textbf{Estimation method:} We use \textbf{matching} on observable
covariates to compare treated and untreated buildings:

\begin{itemize}
\tightlist
\item
  Building characteristics (e.g., living area, number of rooms, building
  age, )
\item
  Neighborhood characteristics (e.g., average income or age, share of
  immigrants, structural features)
\end{itemize}

After successful matching, the treatment effect is estimated through the
following equation \parencite{tchatoka2021}:

\[
\begin{aligned}
\tau_{\text{school}} 
&= \mathbb{E}\!\left[ \ln(P_{1ij}) - \ln(P_{0ij}) \,\middle|\, X_i \right] \\
&= \mathbb{E}\!\left[ \ln(P_{1ij}) \,\middle|\, X_i, D = 1 \right]
  - \mathbb{E}\!\left[ \ln(P_{0ij}) \,\middle|\, X_i, D = 0 \right]
\end{aligned}
\]
\end{frame}

\begin{frame}{Econometric Model (OLS Specification) - First model}
\phantomsection\label{econometric-model-ols-specification---first-model}
We estimate the following log-linear autoregressive hedonic regression
for elementary and secondary schools each \parencite{lu2023}:

\[
\log(\text{P}_i)
= \alpha
+ \beta_1 \, D_i
+ \mathbf{X}_i' \boldsymbol{\gamma}
+ \mathbf{N}_i' \boldsymbol{\delta}
+ \text{FE}_{r(i)}
+ \varepsilon_i
\]

\textbf{Where:}

\begin{itemize}
\tightlist
\item
  log(\(P_i\)): price of a building per \(m^2\)
\item
  \(D_i\): Is within treatment distance to a secondary school (=1) or
  not (=0)
\item
  \(\mathbf{X}_i\): set of building characteristics
\item
  \(\mathbf{N}_i\): set of neighborhood characteristics
\item
  \(\text{FE}_{r(i)}\): Regional fixed effects to account for spatial
  effects at the municipality level
\item
  \(\varepsilon_i\): error term
\end{itemize}
\end{frame}

\begin{frame}{Multiple treatment regime - Second model}
\phantomsection\label{multiple-treatment-regime---second-model}
We are especially interested in the price premium of an available school
that offers academic track compared to other secondary schools
(treatment heterogeneity).

The basic specifiaction is extended referring to a multiple treatment
regime:

\[
\log(\text{P}_i)
= \alpha
+ \beta_1 \, D_i
+ \beta_2 (D_i \times A_i)
+ \mathbf{X}_i' \boldsymbol{\gamma}
+ \mathbf{N}_i' \boldsymbol{\delta}
+ \text{FE}_{r(i)}
+ \varepsilon_i
\] \textbf{Where:}

\begin{itemize}
\tightlist
\item
  \(A_i\): Nearest school offers academic track (=1), otherwise (=0)
\item
  All other variables are the same as for the main specification
\end{itemize}
\end{frame}

\begin{frame}{Challenges and Limitations}
\phantomsection\label{challenges-and-limitations}
\begin{itemize}
\tightlist
\item
  \textbf{Endogeneity:} better schools tend to be located in affluent
  neighborhoods and students endowed with those priviledged backgrounds
  generally achieve higher educational outcomes \parencite{fack2010}
\item
  \textbf{Assumptions}: It may be unrealistic that the capitalization
  effects of upper secondary schools are uniform across the whole space
  \parencite{wen2018}
\item
  \textbf{School access rules:} In some regions, school choice or
  private alternatives may weaken capitalization effects.
\item
  \textbf{Interpretation:} It is arguably that property prices can be
  interpreted as the willingness to pay for amenities
  \parencite{jayantha2015} and therefore the difference between the
  groups as a premium for educational opportunities
\item
  \textbf{Price validity:} Property prices from ImmoScout are
  \textit{asking prices} --- not actual transaction prices. This has
  implications on the interpretations of the results and has to be taken
  into account.
\item
  \textbf{Data:} Incomplete sets of observed building (\(\mathbf{X}_i\))
  and neighborhood (\(\mathbf{N}_i\)) characteristics
\end{itemize}
\end{frame}

\section{Results}\label{results}

\begin{frame}{Descriptives - Price Summary}
\phantomsection\label{descriptives---price-summary}
\textbf{Two Groups}

\begin{table}
\centering\begingroup\fontsize{9}{11}\selectfont

\begin{tabular}{lrrr}
\toprule
\multicolumn{2}{c}{ } & \multicolumn{2}{c}{Price per sqm} \\
\cmidrule(l{3pt}r{3pt}){3-4}
Group & N & Mean & Std. Dev.\\
\midrule
School nearby & 4777 & 3113.02 & 1194.97\\
Control & 13934 & 3238.50 & 1314.23\\
\bottomrule
\end{tabular}
\endgroup{}
\end{table}

\textbf{Three Groups}

\begin{table}
\centering\begingroup\fontsize{9}{11}\selectfont

\begin{tabular}{lrrr}
\toprule
\multicolumn{2}{c}{ } & \multicolumn{2}{c}{Price per sqm} \\
\cmidrule(l{3pt}r{3pt}){3-4}
Group & N & Mean & Std. Dev.\\
\midrule
abitur & 3317 & 3226.07 & 1218.31\\
non\_abitur & 1460 & 2856.20 & 1098.12\\
control & 13934 & 3238.50 & 1314.23\\
\bottomrule
\end{tabular}
\endgroup{}
\end{table}
\end{frame}

\begin{frame}{Descriptives - Housing Characteristics Summary}
\phantomsection\label{descriptives---housing-characteristics-summary}
\textbf{Two Groups}

\begin{table}
\centering\begingroup\fontsize{7}{9}\selectfont

\begin{tabular}{lrrrrrrrr}
\toprule
\multicolumn{1}{c}{ } & \multicolumn{2}{c}{Living Area} & \multicolumn{2}{c}{Site Area} & \multicolumn{2}{c}{Rooms} & \multicolumn{2}{c}{Baths} \\
\cmidrule(l{3pt}r{3pt}){2-3} \cmidrule(l{3pt}r{3pt}){4-5} \cmidrule(l{3pt}r{3pt}){6-7} \cmidrule(l{3pt}r{3pt}){8-9}
Group & Mean & Std. Dev. & Mean & Std. Dev. & Mean & Std. Dev. & Mean & Std. Dev.\\
\midrule
School nearby & 173.74 & 68.53 & 591.08 & 395.79 & 6.08 & 2.59 & 1.92 & 1.13\\
Control & 180.67 & 77.43 & 623.50 & 474.98 & 6.30 & 2.97 & 2.06 & 1.37\\
\bottomrule
\end{tabular}
\endgroup{}
\end{table}

\textbf{Three Groups}

\begin{table}
\centering\begingroup\fontsize{6}{8}\selectfont

\begin{tabular}{lrrrrrrrr}
\toprule
\multicolumn{1}{c}{ } & \multicolumn{2}{c}{Living Area} & \multicolumn{2}{c}{Site Area} & \multicolumn{2}{c}{Rooms} & \multicolumn{2}{c}{Baths} \\
\cmidrule(l{3pt}r{3pt}){2-3} \cmidrule(l{3pt}r{3pt}){4-5} \cmidrule(l{3pt}r{3pt}){6-7} \cmidrule(l{3pt}r{3pt}){8-9}
Group & Mean & Std. Dev. & Mean & Std. Dev. & Mean & Std. Dev. & Mean & Std. Dev.\\
\midrule
abitur & 172.08 & 67.55 & 577.79 & 395.65 & 6.05 & 2.61 & 1.91 & 1.10\\
non\_abitur & 177.53 & 70.57 & 621.25 & 394.57 & 6.14 & 2.55 & 1.95 & 1.19\\
control & 180.67 & 77.43 & 623.50 & 474.98 & 6.30 & 2.97 & 2.06 & 1.37\\
\bottomrule
\end{tabular}
\endgroup{}
\end{table}
\end{frame}

\begin{frame}{Descriptives - Balance Test}
\phantomsection\label{descriptives---balance-test}
\textbf{School nearby vs.~Control}

\begin{table}
\centering\begingroup\fontsize{8}{10}\selectfont

\begin{tabular}{lrrrrl}
\toprule
Variable & Mean Treatment & Mean Control & Difference & t-statistic & p-value\\
\midrule
price\_sqm & 3238.498 & 3113.023 & 125.475 & 6.102 & 0.000\\
living\_area & 180.673 & 173.744 & 6.929 & 5.828 & 0.000\\
site\_area & 623.503 & 591.075 & 32.428 & 4.633 & 0.000\\
rooms\_n & 6.303 & 6.080 & 0.223 & 4.935 & 0.000\\
baths\_n & 2.060 & 1.919 & 0.140 & 6.991 & 0.000\\
age\_building & 41.627 & 37.911 & 3.715 & 5.478 & 0.000\\
\bottomrule
\end{tabular}
\endgroup{}
\end{table}
\end{frame}

\begin{frame}{Descriptives - Boxplot}
\phantomsection\label{descriptives---boxplot}
\begin{center}
\includegraphics[width=0.8\linewidth,height=\textheight,keepaspectratio]{presentation_files/figure-beamer/unnamed-chunk-6-1.pdf}
\end{center}
\end{frame}

\begin{frame}{Main specification}
\phantomsection\label{main-specification}
\begin{table}
\centering
\begin{talltblr}[         %% tabularray outer open
caption={Effect of Secondary School Proximity on House Prices},
note{}={+ p \num{< 0.1}, * p \num{< 0.05}, ** p \num{< 0.01}, *** p \num{< 0.001}},
note{ }={Note: Robust standard errors clustered at the municipality level in parentheses.},
]                     %% tabularray outer close
{                     %% tabularray inner open
colspec={Q[]Q[]Q[]Q[]Q[]Q[]Q[]},
hline{2}={1-7}{solid, black, 0.05em},
hline{4}={1-7}{solid, black, 0.05em},
hline{1}={1-7}{solid, black, 0.1em},
hline{9}={1-7}{solid, black, 0.1em},
column{2-7}={}{halign=c},
column{1}={}{halign=l},
}                     %% tabularray inner close
& (1) & (2) & (3) & (1)  & (2)  & (3)  \\
School nearby & \num{0.015} & \num{0.006} & \num{0.005} & \num{0.003} & \num{0.010} & \num{0.010} \\
& (\num{0.012}) & (\num{0.009}) & (\num{0.009}) & (\num{0.013}) & (\num{0.010}) & (\num{0.011}) \\
Observations & \num{18703} & \num{18703} & \num{18703} & \num{9537} & \num{9537} & \num{9537} \\
R² & \num{0.397} & \num{0.614} & \num{0.617} & \num{0.415} & \num{0.626} & \num{0.629} \\
Building controls & - & Yes & Yes & - & Yes & Yes \\
Neighborhood controls & - & - & Yes & - & - & Yes \\
Region fixed effects & Yes & Yes & Yes & Yes & Yes & Yes \\
\end{talltblr}
\end{table}
\end{frame}

\begin{frame}{Heterogeneity Model}
\phantomsection\label{heterogeneity-model}
\begin{table}
\centering
\begin{talltblr}[         %% tabularray outer open
caption={Effect of School with academic track proximity on House Prices},
note{}={* p \num{< 0.1}, ** p \num{< 0.05}, *** p \num{< 0.01}},
note{ }={Note: Robust standard errors clustered at the municipality level in parentheses.},
]                     %% tabularray outer close
{                     %% tabularray inner open
colspec={Q[]Q[]Q[]},
hline{2}={1-3}{solid, black, 0.05em},
hline{6}={1-3}{solid, black, 0.05em},
hline{1}={1-3}{solid, black, 0.1em},
hline{11}={1-3}{solid, black, 0.1em},
column{2-3}={}{halign=c},
column{1}={}{halign=l},
}                     %% tabularray inner close
& Full (U) & Full (M) \\
School nearby & \num{-0.013} & \num{-0.009} \\
& (\num{0.015}) & (\num{0.015}) \\
School nearby × Gymnasium nearby & \num{0.026} & \num{0.027} \\
& (\num{0.016}) & (\num{0.017}) \\
Observations & \num{18703} & \num{9537} \\
R² & \num{0.617} & \num{0.629} \\
Building controls & Yes & Yes \\
Neighborhood controls & Yes & Yes \\
Region fixed effects & Yes & Yes \\
\end{talltblr}
\end{table}
\end{frame}

\section{Policy Implications and further
research}\label{policy-implications-and-further-research}

\section{Appendix}\label{appendix}

\begin{frame}{Controls (1/2)}
\phantomsection\label{controls-12}
\begin{longtable}[]{@{}ll@{}}
\caption{Property characteristics}\tabularnewline
\toprule\noalign{}
Variable & Description \\
\midrule\noalign{}
\endfirsthead
\toprule\noalign{}
Variable & Description \\
\midrule\noalign{}
\endhead
living\_area & Living area of the dwelling (m²) \\
site\_area & Site area of the dwelling (m²) \\
rooms\_n & Number of rooms \\
baths\_n & Number of baths \\
age\_building & Age of the building (years) \\
cellar & Dummy for presence/absence of a cellar \\
\bottomrule\noalign{}
\end{longtable}
\end{frame}

\begin{frame}{Controls (1/2)}
\phantomsection\label{controls-12-1}
\begin{longtable}[]{@{}
  >{\raggedright\arraybackslash}p{(\linewidth - 2\tabcolsep) * \real{0.2128}}
  >{\raggedright\arraybackslash}p{(\linewidth - 2\tabcolsep) * \real{0.7872}}@{}}
\caption{Neighborhood characteristics}\tabularnewline
\toprule\noalign{}
\begin{minipage}[b]{\linewidth}\raggedright
Variable
\end{minipage} & \begin{minipage}[b]{\linewidth}\raggedright
Description
\end{minipage} \\
\midrule\noalign{}
\endfirsthead
\toprule\noalign{}
\begin{minipage}[b]{\linewidth}\raggedright
Variable
\end{minipage} & \begin{minipage}[b]{\linewidth}\raggedright
Description
\end{minipage} \\
\midrule\noalign{}
\endhead
immigrants\_percents & Share of immigrants in the 1 km² grid (\%) \\
average\_age & Average age of residents in the 1km² grid (years) \\
pharmacy & (=1) if the grid-cell contains at least 1 pharmacy, (=0)
otherwise \\
hospital & (=1) if the grid-cell contains at least 1 hospital, (=0)
otherwise \\
doctors & (=1) if the grid-cell contains at least 1 doctor's office,
(=0) otherwise \\
park & (=1) if the grid-cell contains at least 1 park, (=0) otherwise \\
\bottomrule\noalign{}
\end{longtable}
\end{frame}


\begin{frame}[allowframebreaks]{Literature}
  \bibliographytrue
  \printbibliography[heading=none]
\end{frame}



\end{document}
